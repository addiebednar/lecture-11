\documentclass{tufte-handout}

\usepackage{xcolor}

% set hyperlink attributes
\hypersetup{colorlinks}

% set image attributes:
\usepackage{graphicx}
\graphicspath{ {images/} }

% indent subsections
\newenvironment{subs}
  {\adjustwidth{3em}{0pt}}
  {\endadjustwidth}
  
% ============================================================

% define the title
\title{SOC 4015/5050: Lecture 11 Functions}
\author{Christopher Prener, Ph.D.}
\date{Fall 2018}
% ============================================================
\begin{document}
% ============================================================
\maketitle % generates the title
% ============================================================

\vspace{5mm}
\section{Packages}
\begin{itemize}
\item \texttt{base}
\item \texttt{ggplot2}
\item \texttt{ggstatsplot}
\item \texttt{grdevices}
\item \texttt{Hmisc}
\item \texttt{naniar}
\item \texttt{pwr}
\item \texttt{stats}
\end{itemize}

\vspace{5mm}
\section{In-line \texttt{R} Code}
\begin{subs}
\subsection{Rounding Values}
\noindent \texttt{base::}{\color{red}\texttt{round}}\texttt{(\textit{x}, digits = \textit{val})}

\vspace{3mm}
\subsection{Implementing In-line \texttt{R} Code}
\begin{verbatim}
```{r load-data}
library(ggplot2)
auto <- mpg
```

The average highway fuel efficiency in the data set is 
`r round(mean(auto$hwy), digits = 3)`.
\end{verbatim}
\end{subs}

\vspace{5mm}
\section{Scatterplots}
\begin{subs}
\subsection{Basic Scatterplot}
\noindent \texttt{ggplot2::}{\color{red}\texttt{geom\_point}}\texttt{(mapping = aes(x = \textit{xvar}, y = \textit{yvar}))}

\vspace{3mm}
\subsection{Scatterplot with Smoothed Line}
\noindent \texttt{ggplot2::}{\color{red}\texttt{geom\_smooth}}\texttt{(method = "lm", \\mapping = aes(color = "\#\textit{hex}"))}\sidenote{Use \href{http://colorhexa.com}{http://colorhexa.com} to selection hex values!}

\vspace{3mm}
\subsection{Scatterplot with Grouping Variable}
\noindent \texttt{ggplot2::}{\color{red}\texttt{geom\_point}}\texttt{(mapping = aes(x = \textit{xvar}, y = \textit{yvar}, \\ color = \textit{groupVar}))}

\vspace{3mm}
\subsection{Scatterplot with Smoothed Line by Grouping Variable}
\noindent \texttt{ggplot2::}{\color{red}\texttt{geom\_smooth}}\texttt{(method = "lm", \\mapping = aes(color = \textit{groupVar}, linetype = \textit{groupVar}))}

\vspace{3mm}
\subsection{Scatterplot with Facet}
\noindent \texttt{ggplot2::}{\color{red}\texttt{facet\_grid}}\texttt{(. \textasciitilde\ \textit{facetVar})}

\vspace{3mm}
\subsection{Statistical Scatterplot}
\noindent \texttt{ggstatsplot::}{\color{red}\texttt{ggscatterstats}}\texttt{(data = \textit{data}, x = \textit{xvar}, \\y = \textit{yvar})}

\vspace{3mm}
\subsection{Create a File to Save a Statistical Scatterplot}
\noindent \texttt{grdevices::}{\color{red}\texttt{png}}\texttt{(\textit{filepath}, width = \textit{val}, height = \textit{val})}\sidenote{Values for \texttt{width} and \texttt{height} should be specified in points.}

\vspace{3mm}
\subsection{Writing the Statistical Scatterplot to File}
\noindent \texttt{grdevices::}{\color{red}\texttt{dev.off}}\texttt{()}
\end{subs}

\vspace{5mm}
\section{Creating Data Objects}
\begin{subs}
\subsection{Creating a Vector}
\noindent \texttt{base::}{\color{red}\texttt{c}}\texttt{(\textit{element, element, element})}

\vspace{3mm}
\subsection{Creating a Data Frame}
\noindent \texttt{base::}{\color{red}\texttt{c}}\texttt{(\textit{vector, vector}, stringsAsFactors = FALSE)}

\vspace{3mm}
\subsection{Creating a Matrix}
\noindent \texttt{base::}{\color{red}\texttt{as.matrix}}\texttt{(\textit{objectName})}
\end{subs}

\vspace{5mm}
\section{Missing Data}
\begin{subs}
\subsection{Missing Data Anlysis}
\noindent \texttt{naniar::}{\color{red}\texttt{miss\_var\_summary}}\texttt{(\textit{data})}

\vspace{3mm}
\subsection{Remove All Missing Data}
\noindent \texttt{stats::}{\color{red}\texttt{na.omit}}\texttt{(\textit{data})}
\end{subs}

\vspace{5mm}
\section{Pearson's r}
\begin{subs}
\subsection{Basic Approach}
\noindent \texttt{stats::}{\color{red}\texttt{cor}}\texttt{(\textit{dataFrame}, \textit{use}, method = "pearson")}

\vspace{3mm}
\subsection{\texttt{Hmisc} Approach}
\noindent \texttt{Hmisc::}{\color{red}\texttt{rcorr}}\texttt{(\textit{matrix}, type = "pearson")}

\vspace{3mm}
\subsection{Full Table Approach}
\noindent {\color{red}\texttt{corrTable}}\texttt{(\textit{dataFrame}, coef = "pearson", listwise = TRUE, \\ round = 3, pStar = 3, ...)}
\end{subs}

\vspace{5mm}
\section{Sample Size Estimate}
\noindent \texttt{pwr::}{\color{red}\texttt{pwr.r.test}}\texttt{(r = \textit{rVal}, sig.level = .05, power = \textit{powerVal}, \\ alernative = "two.sided")}

% ============================================================
\end{document}